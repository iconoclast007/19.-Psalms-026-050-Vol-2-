\section{Psalm 28 Outlines}

\subsection{My Outlines}

\subsubsection{David's Very Personal Faith}
%\textbf{Introduction:} Psalm 28:\footnote{04 July 2016, Keith Anthony.}
\index[speaker]{Keith Anthony!Psalm 28 (David's Very Personal Faith)}
\index[series]{Psalms (Keith Anthony)!Psalm 28 (David's Very Personal Faith)}
\index[date]{2016/07/04!Psalm 28 (David's Very Personal Faith) (Keith Anthony)}
\begin{compactenum}[I.]
    \item His \textbf{Supplications} \index[scripture]{Psalms!Psa 028:02}\index[scripture]{Psalms!Psa 028:06}(Psa 28:2, 6)
    \item His \textbf{Surety} \index[scripture]{Psalms!Psa 028:05}(Psa 28:5)
    \item His \textbf{Song} \index[scripture]{Psalms!Psa 028:07}(Psa 28:7)
    \item His \textbf{Strength} \index[scripture]{Psalms!Psa 028:07}\index[scripture]{Psalms!Psa 028:08}(Psa 28:7, 8)
    \item His \textbf{Shield} \index[scripture]{Psalms!Psa 028:07}(Psa 28:7)
    \item His \textbf{Source} of Rejoicing \index[scripture]{Psalms!Psa 028:07}(Psa 28:7)
    \item His \textbf{Salvation} \index[scripture]{Psalms!Psa 028:08}\index[scripture]{Psalms!Psa 028:09}(Psa 28:8, 9)
\end{compactenum}

%Psalm 28 Clark Herring, 10/23/18, Fundamental Baptist Sermon Outlines
%“What to Remember in the Silent Times”
%I. Remember to Recognize the Rock (vs1-2)
%A) The Distinction of the Rock (vs1a) (my)
%B) The Description of the Rock (vs1b) (Rock)
%C) The Dependence on the Rock (vs1c)
%D) The Devotion to the Rock (vs2)
%II. Remember the Results of the Rebellious (vs3-5)
%A) Their Luring (vs3)
%B) Their Labor (vs4)
%C) Their Leader (vs5)
%III. Remember to Rejoice with the Righteous (vs6-9)
%A) The Attention of our Lord (vs6)
%B) The Attributes of our Lord (vs7-8)
%C) The Abilities of our Lord (vs9)
%a) to Save
%b) to Satisfy
%c) to Sustain
%d) to Supply
%e) to Strengthen

%Psalm 28 Clarence Billheimer, 1/22/20, Fundamental Baptist Sermon Outlines
%Introduction: Silence is an eerie thing. You know how you feel when you’re talking to someone and they sit there silent, appearing as though they’re not listening to you. You know how frustrating it is to call for someone and no one answers. Imagine what it can be like when God appears to be silent! After all, He’s omniscient, omnipresent, and omnipotent. He’s able to hear millions of voices at one and not be confused. He can respond to cries instantly and many at once. In Psalm 28 there’s a prayer of David he may have prayed while being pursued by Saul or some other enemy. Believe me, when we’re in trouble of any kind, that’s not the time we want to feel God is silent! Let’s follow the progression of this experience in David’s life to see the Lord was not silent at all.
%I. The cry (28:1, 2)
%A. To a strong source
%B. With a spoken supplication
%C. Toward a specific spot
%D. Because of a potential sad slide
%II. The condemnation (28:3-5)
%A. The wicked and workers of iniquity are deceitful.
%B. The wicked and workers of iniquity are doomed.
%C. The wicked and workers of iniquity will be destroyed.
%III. The chorus (28:6-9)
%A. God has heard.
%B. Man’s heart has trusted and been helped.
%C. Man’s response is heralded.
%D. God’s abilities are echoed toward others.
%Conclusion: The final verse closes this prayer of David with a plea that God will do for others what He had just done for David. That should be our prayer when we experience the hearing of God and answer to prayer.