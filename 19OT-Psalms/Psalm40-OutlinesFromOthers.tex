\subsection{Outlines from Others}

\subsubsection{Principles Pertaining To The Pit}
\index[speaker]{Michael Lovett!Psalm 040 (Principles Pertaining To The Pit)}
\index[series]{Psalms (Michael Lovett)!Psalm 040 (Principles Pertaining To The Pit)}
\index[date]{2014/1103!Psalm 040 (Principles Pertaining To The Pit) (Michael Lovett)}
Probably one of the Psalms  quoted frequently when preaching would be the one we have before us this morning. It is one of many passages that I truly cherish. It is and was such a blessing to me when I first came across it. I realized in many aspects that the truth contained within that verse was in many aspects what God did for me. You have heard me mention on more than one occasion that God picked me up out of an horrible pit, and He placed my feet upon a rock and He has established my goings…… and I say Thank you and Praise the Lord He did.

If you are not careful in life many times we can find ourselves in a place we would much rather not be. Most folks never intend for it to happen, but it does. They want to be in a better place, but many times “the Want To” gets overshadowed by “the Will To” and that is usually a result of a particular place that we have settled into. We call that place in today’s vernacular “a rut” however the Bible refers to it many times as a “pit.” For some they have either chosen to stay in their Pit or drag their Pit around with them and relish in the fact they have one.

As a Pastor I have come to realize that way too many folks live most of their life in a Pit. God never intended for us to reside in a Pit and if He so chooses for you to be in one, for a season (Joseph, Jeremiah) then He will inform you of the reason why, and the duration.

In my few years in this world I have come to realize that Pits come in all shapes and sizes. I have as well realized that there is no discrimination from the Pit, as to its inhabitants….it welcomes all dwellers. As I reflect in my mind in the Scriptures I find that many good folk as well as bad have spent time in “the Pit”. Some grow use to it while others long for and desire to get out of it. I want to briefly look at just a few that come to mind by way of introduction to set the groundwork for the thought for this morning. Pits I find in Scripture that comes to mind are:\footnote{03 November 2014, Michael Lovett, as posted in [BaptistOutlineSharing] }
\begin{compactenum}[I.][9]
    \item \textbf{The PIT of SENTENCING} - Cain - Gen 4:13 ``And Cain said unto the LORD, My punishment is greater than I can bear.''
    \item \textbf{The PIT of SPITE} - Joseph - Gen 37:18-20 ``And when they saw him afar off, even before he came near unto them, they conspired against him to slay him. 19 And they said one to another, Behold, this dreamer cometh. 20 Come now therefore, and let us slay him, and cast him into some pit, and we will say, Some evil beast hath devoured him: and we shall see what will become of his dreams.''
    \item \textbf{THE PIT of SANITIZING} – Joseph – Gen 50:19 ``And Joseph said unto them, Fear not: for am I in the place of God? 20 But as for you, ye thought evil against me; but God meant it unto good, to bring to pass, as it is this day, to save much people alive.''
    \item \textbf{The PIT of SORROW} – Jacob - Gen37:34 ``And Jacob rent his clothes, and put sackcloth upon his loins, and mourned for his son many days.''
    \item \textbf{The PIT of SULKING} - Elijah - 1 Kings 19:1-4 ``And Ahab told Jezebel all that Elijah had done, and withal how he had slain all the prophets with the sword. 2 Then Jezebel sent a messenger unto Elijah, saying, So let the gods do to me, and more also, if I make not thy life as the life of one of them by to morrow about this time. 3 And when he saw that, he arose, and went for his life, and came to Beersheba, which belongeth to Judah, and left his servant there. 4 But he himself went a day's journey into the wilderness, and came and sat down under a juniper tree: and he requested for himself that he might die; and said, It is enough; now, O LORD, take away my life; for I am not better than my fathers.'' 
    \item \textbf{The PIT of STUBBORNESS} - Prov7:11 “And, behold, there met him a woman with the attire of an harlot, and subtil of heart. 11 She is loud and stubborn; her feet abide not in her house:” Duet21:18-21 “If a man have a stubborn and rebellious son, which will not obey the voice of his father, or the voice of his mother, and that, when they have chastened him, will not hearken unto them: 19 Then shall his father and his mother lay hold on him, and bring him out unto the elders of his city, and unto the gate of his place; 20 And they shall say unto the elders of his city, This our son is stubborn and rebellious, he will not obey our voice; he is a glutton, and a drunkard. 21 And all the men of his city shall stone him with stones, that he die: so shalt thou put evil away from among you; and all Israel shall hear, and fear.”
    \item \textbf{The PIT of SEPARATION} – Jeremiah - Jer 38:6 ``Then took they Jeremiah, and cast him into the dungeon of Malchiah the son of Hammelech, that was in the court of the prison: and they let down Jeremiah with cords. And in the dungeon there was no water, but mire: so Jeremiah sunk in the mire.''
    \item \textbf{The PIT of SELFISHNESS} - Gehazi - 2 Kings 5:20 ``But Gehazi, the servant of Elisha the man of God, said, Behold, my master hath spared Naaman this Syrian, in not receiving at his hands that which he brought: but, as the LORD liveth, I will run after him, and take somewhat of him.'' (ended up with Leprosy)
    \item \textbf{The PIT of SIN} - David - 2 Sam 11: (Bathsheba, Uriah, the Child)
\end{compactenum}
So we see, with only a few of the dozens of Pits that can be found in Scripture, that any kind of Person can find themselves in a Pit. However, God didn't come so that we could \emph{Pout in our Pit} but rather \emph{Praise our Pit}. With that being said I want to look at some Aspects of how to get out of your Pit, preaching on this thought:
\begin{compactenum}[I.]
    \item \textbf{THE PERSON OF THE PIT} All of us, if we are not careful, can end up in a Pit….Sinner and Saint alike, yet it all comes down to one thing and that is the WANT TO of the one in the Pit. In order to come out successfully he or she MUST Acknowledge, Accept and Admit the Purpose and Principle of the Pit. Are you in a Pit today? Oh, you say it’s just a small one - don’t worry it will get bigger if you don’t get out now. Decide today to not be the Person of a Pit.  (verses 1-3) 
    \begin{compactenum}
        \item WE SEE HIS ACKNOWLEDGEMENT (verse 1a
        \item WE SEE HIS ACCEPTANCE (verse 1b)
        \item WE SEE HIS ADMISSION (verse 1c)
    \end{compactenum}
    \item \textbf{THE PROBLEM OF THE PIT} If you’re not careful the Condition, the Constraint and the Control of the Pit become such a Problem that getting out is near next to impossible. I have seen many people in my life that make ``their Pit'' a badge of honor. You begin to justify it and say ``oh well it’s just my lot in life''. I have joked many times with folks and said the first step is in admitting it, but it is. Recognize today the Problem of the Pit and begin the ascent out. Are you in a Pit today? If you’re not stay out of one, if you are decide today to begin to Part from it. (verse 12)
    \begin{compactenum}
        \item WE SEE THE CONDITION OF THE PIT (verse 12a)
        \item WE SEE THE CONSTRAINT OF THE PIT (verse 12b)
        \item WE SEE THE CONTROL OF THE PIT (verse 12c)
    \end{compactenum}
    \item \textbf{THE PLEA OF THE PIT} I think most of the time we end up in ``Our Pits'' because we fail to see that HE is the One that keeps us out of them and not we ourselves. He allows us to get into them or places us in them, to see if we will ``wait patiently and cry unto Him.'' To begin the ascent out there must be a Plea and it all begins with a Declaration, then a Delight to do His will and then a recognition that He is the Deliverer of our Pit and not we ourselves.  What is your Plea when you find yourself in a Pit? I find a lot of folks that just seem to like there Pit in their own silly little way.  (verses 5, 13, 13, 17)
    \begin{compactenum}
        \item THERE IS A DECLARATION (verse 5)
        \item THERE IS A DELIGHT (verse 8)
        \item THERE IS A DELIVERANCE (verses 13, 17)
    \end{compactenum}
    \item \textbf{THE PRODUCT OF THE PIT} You can either make the Pit a Burden or a Blessing. If I look back there have been a lot of Pits that I have found myself in - if I chose to stay there they could have easily become a burden, however I chose to make them a blessing because of The Person that Pulled me out. If you were to examine your life today is there any Pits you’re in? Are they just a little low space now, or have they now turned into a valley? They can easily become a huge Pit unless you decide today to Part from your Pit and set your feet upon The ROCK.  (verses 3, 4) 
    \begin{compactenum}
            \item A NEW SONG (verse 3a)
            \item A NEW STANDARD (verse 3b)
            \item A NEW STANDING (verse 4a)
    \end{compactenum}
\end{compactenum}

