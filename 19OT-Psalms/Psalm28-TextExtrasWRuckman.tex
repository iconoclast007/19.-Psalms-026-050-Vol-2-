\chapter{Psalm 29}

\footnote{In this psalm we have the the Lord identified a total of 18 times. We see a parallel of all of history in verses 3 through 9, with the voice of God upon the waters in verse 3, and finishing with God in his Temple in verse 9. In verse, the third of the seven verses, we see the cedars of Lebanon being broken. Could this correspond to the cedars in Solomon's Temple being destroyed as Israel gets conquered. IN verse 8, the wilderness is shaken. Could the word ``wilderness'' correspond the the Jews in the wilderness in the Great Tribulation? In verse 5, the word ``skip'' is used the only time in scripture with the cedars skipping like a calf. In the presence of the Lord (Isaiah 55:12) the trees of the field will ``clap'' their hands.}
[3] \footnote{These are the “seven thunders” of Revelation 10:4. ``The God of glory thundereth,'' and when He thunders a rapture takes place (Job 37:1–-4). His voice is ``in the thunder'' (Job 40:9, John 12:29) when the ``trumpet sounds within a man’s soul''” (“ah ain’t got long to be hyere”). This is the ``trump of God'' (1 Thessalonains 4:16) for it is a voice like ``a trumpet'' (Revelation 4:1–-2) which will be mistaken for ``thunder'' (John 12:29). It is not the seventh trumpet blown by an angel in Revelation 11:15. We do not know what the seven thunders in Revelation said, but the expression ``the voice of the Lord'' occurs exactly seven times in this passage (verses 3, 4 (2x), 5, 7, 8, 9). It is absolutely certain that no Christian in the Body of Christ is present in Hebrews 3:14, 6:4–-8, 10:27–-30; Revelation 11:1–-12, 14:12; or Matthew 24:12–-20. Possibilities are \cite{Ruckman1992Psalms}: }    \footnote{\href{http://www.devotional.net/uploads/147/95775.pdf}{Cheatham} has suggested that these seven raptures are: \begin{compactenum}
    \item Enoch (Genesis 5:21-23 and Hebrews 11:5-6)
    \item Elijah (2 Kings 2:1-11)
    \item Jesus (Matthew 28:6-7 and Acts 1:9-11)
    \item Church (1 Thessalonians 4:13-18, 1 Corinthians 15:51-58, Revelation 4:1-4)
    \item Mid-Tribulation Saints (Revelation 7:9-17)
    \item 144,000 Jews (Revelation 14:1-5)
    \item Two Witnesses (Revelation 11:1-12)
\end{compactenum} }  \footnote{See the parallel in Genesis 6:12--22, where God speaks to Noah, seven times.}
[4] 
[10] \footnote{Verse 10 sacks all of the commentators in one sack. It was a reference to the “waters” of verse 3, which were there in Genesis 1:2–3 and are mentioned in Habakkuk 3:10. But many years ago all of the unbelieving critics who believed in “plenary, verbally inspired, infallible, inerrant, original frisbees” gave way to the Devil. Psalm 93 points to the “waters that be above the heavens” (see Ps. 148:4). “Vapour” is not even to be considered, for “vapour” is a separate item from the waters above the heavens (Ps. 148:4, 7).  \cite{Ruckman1992Psalms}}

