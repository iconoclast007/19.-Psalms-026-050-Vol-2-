\section{Psalm 28 Comments}

\subsection{Numeric Nuggets}
The 13-letter word ``supplications'' is used twice in Psalm 28. There are 13 words in  verse 6.

\subsection{Introduction}
Phillips suggests that the Psalm may have been composed at at a time of  national crisis for Israel, specifically the period of civil war led by Absalom (2 Samuel 17-18). Verse 4 and 5 are imprecatory, calling for judgment against the evildoers, likely Ahithophel and his followers. The fact that Absalom was among this group was a source of immense distress given David's deep love for Absalom. Indeed, in the incident we see Absalom picturing rebellious Isreal rejecting God, seen in David. In Ahothophel, we see the great instigator himself, possibly referenced by ``the wicked'' in verse  3 (see cross references to ``wicked'' such as in Psalm 10). David also pictures the trusting remnant of Israel, redeemed in the end. The allusion is seen in the return of the plural pronouns in verses 8 and 9.  \cite{Phillips2001ExploringPsalms1} 

\subsection{Psalm 28:2}
An oracle is a thing or a place (in this case Jerusalem and the Temple), and oracles are the decrees that come from such. An example of the use of ``oracle'' can be seen in  2 Samuel 16:23.\footnote{\textbf{2 Samuel 16:23} - And the counsel of Ahithophel, which he counselled in those days, was as if a man had enquired at the oracle of God: so was all the counsel of Ahithophel both with David and with Absalom.} The oracles of God are described in Acts 7:38, Romans 3:2, Hebrews 5:12, and 1 Peter 4:11.\footnote{\textbf{Acts 7:38} - This is he, that was in the church in the wilderness with the angel which spake to him in the mount Sina, and with our fathers: who received the lively oracles to give unto us:}\footnote{\textbf{Romans 3:2} -  Much every way: chiefly, because that unto them were committed the oracles of God.}\footnote{\textbf{Hebrews 5:12} - or when for the time ye ought to be teachers, ye have need that one teach you again which be the first principles of the oracles of God; and are become such as have need of milk, and not of strong meat.}\footnote{\textbf{1 Peter 4:11} - If any man speak, let him speak as the oracles of God; if any man minister, let him do it as of the ability which God giveth: that God in all things may be glorified through Jesus Christ, to whom be praise and dominion for ever and ever. Amen.}

\subsection{Psalm 28:3}
Compare this prayer with that in Psalm 26:9.\footnote{\textbf{Psalm 26:9} - Gather not my soul with sinners, nor my life with bloody men:} 

\subsection{Psalm 28:8}
His ``anointed'' in the verse is a reference to David historically, and Jesus Christ doctrinally.\cite{Ruckman1992Psalms}

\subsection{Psalm 28:9}
There are four specific requests in verse 9, all aimed at the Jews:\cite{Ruckman1992Psalms}
\begin{compactenum}
	\item ``Save thy people'' (Romans 11:26)\footnote{\textbf{Romans 11:26} - And so all Israel shall be saved: as it is written, There shall come out of Sion the Deliverer, and shall turn away ungodliness from Jacob:}
	\item ``bless thin inheitance'' (Psalm 115:12)\footnote{\textbf{Psalm 115:12} - The LORD hath been mindful of us: he will bless us; he will bless the house of Israel; he will bless the house of Aaron.}\footnote{\textbf{Isaiah 2:6} - }
	\item ``feed them also'' (Ezekiel 34:14)\footnote{\textbf{Exekiel 34:14} - I will feed them in a good pasture, and upon the high mountains of Israel shall their fold be: there shall they lie in a good fold, and in a fat pasture shall they feed upon the mountains of Israel.}
	\item ``lift them up forever'' (Isaiah 65-66, Revelation 21-22)
\end{compactenum}
