\chapter{Psalm 37}
\footnote{\textcolor[rgb]{0.00,0.25,0.00}{\hyperlink{TOC}{Return to end of Table of Contents.}}}\textcolor[rgb]{0.00,0.00,1.00}{\emph{A Psalm} of David.}\\
\\


[2] \footnote{Psalm 37 is one of the most practical Psalms in the collection. It has as much spiritual and devotional material in it (for any saint of any age) as any other Psalm. Many times prophetic subjects and doctrinal matters arise (see vss. 2, 6, 9, 10, 11, 13, 14, 17, 19, 20, 22, 29, 34, 35, 38, and 39), but among these only six of them deal with a New Testament theme: “inheriting the Kingdom” (see Col. 3:24 and Eph. 5:5). Note the matter of the “inheritance” in verses 9, 11, 18, 22, and 29. Notice also, how verses 11 and 29 dovetail into the famous “Sermon on the Mount,” clearly identifying it as a Jewish passage to a nation about to inherit a literal, visible, political, Messianic kingdom. Advent passages are found in verses 2, 6, 10, 13, 20, and 36. You can expect the faculty and staff at Wheaton, Fuller, Moody, Crozier, Union, Colgate, Bob Jones, Kings College, Denver, Dallas, Bible Baptist College, and Alexandria, Egypt, to miss ALL the references everytime they show up. Occupation with “the original text” (Kutilek, Wuest, Sumner, Zodhiates) produces a blind stupidity that is absolutely staggering. \cite{Ruckman1992Psalms}} \footnote{See the use of the word ``grass'' in scripture, especially in James and Revelation.}

[10] \footnote{Observe that remarkable date of the Advent given again as “a little while” (vs. 10). John records it seven times in less than five verses (John 16:16-–19) one time for each year in Daniel’s “Seventieth Week.” In all, the phrase is found 21 times in 16 verses: (1) Job 24:24, (2) Psa 37:10, (3) Isa 63:18, (4) Jer 51:33, (5) Hos 1:4, (6) Hag 2:6, (7) Luk 22:58, (8) Jhn 7:33, (9) Jhn 12:35, (10)  Jhn 13:33, (11) Jhn 14:19, (12) Jhn 16:16, (13) Jhn 16:17, (14) Jhn 16:18, (15) Jhn 16:19, and (16) Heb 10:37.  The phrase ``very little while'' is also found in Isaiah 10:25 and Isaiah 29:17 \cite{Ruckman1992Psalms}}
[11] 
[13] \footnote{The coming “day” of verse 13 is the day “that shall burn as an oven” (Mal. 4:1–3). \cite{Ruckman1992Psalms}}
[14] \footnote{Observe “the poor and needy” again in verse 14 (see comments under Ps. 9:18 and 10:2, 9). Note again that the wicked have “riches” (see James 2:5 and Luke 16:19). It is taken for granted that the “righteous man” has only a “little.” There cannot be any leeway of interpretation here at all, for the bad right arm of “the son of perdition” shows up in the next verse (vs. 17); ALL the commentators and scholars miss it. Kroll’s comment is a classic: ignoring Jeremiah 48:25 and Zechariah 11:17 and Psalm 10:15 and Ezekiel 30:24, the old Lynchburg “champion for Christ” says ``He has a way of making incapable men out of implacable men.'' \cite{Ruckman1992Psalms}} \footnote{Observe the “teeth” again, as in Psalm 35:16, which see. \cite{Ruckman1992Psalms}}


[23] \footnote{Inspirationally, we may note that the steps of a Christian are “ordered” by the Lord in that these steps are commanded, in the sense of a request (or “order”) sent in to be fulfilled and in the sense that disorder is not the Lord’s order for your life. The common twentieth century sign “Out of Order” should not be hanging on a Christian’s life or testimony (vs. 13). Verse 21 is really some verse for a Christian who owes money (see Rom. 13:8). Notice that the “righteous” in verse 21 do not even loan; they give (see Matt. 25:35). Verse 24 can be applied spiritually to the believer in the matter of eternal security, although the context is not the new birth. We are not “upheld by God’s hand”; we are part of God’s hand (see the difference stated in John 10:28 with 1 Cor. 12:12–24). \cite{Ruckman1992Psalms}}

[27] \footnote{``Depart from evil and do good'' was the theme of nearly every sermon that Sam Jones (1847--–1906) ever preached: ``cease to do evil, learn to do well'' (Isa. 1:16--17). Paul says, ``Abhor that which is evil, cleave to that which is good'' (Rom. 12:9). \cite{Ruckman1992Psalms}}
[28] \footnote{Again the eternal security of the New Testament believer is hinted at in verse 28, “they are preserved forever,” but the context is Tribulation saints who “endure to the end” (Matt. 24:13). \cite{Ruckman1992Psalms}}
[29] \footnote{Verse 29 is, again, a reference to the Millennial inheritance which is directly promised six times in the Psalm (vss. 9, 11, 18, 22, 29, 34) and indirectly a seventh time indirectly (vs. 3). The inheritors of this kingdom in the Psalm are given as those who \cite{Ruckman1992Psalms}:
\begin{compactenum}
\item Are meek.
\item Wait on the Lord.
\item Keep His way.
\item Are righteous.
\item Are blessed of Him.
\end{compactenum}}
[30] \footnote{The “righteous” are always “judgmental,” to use the hackneyed News Media cliche. They talk about judgment on queers for getting AIDS. They talk about judgment on America for voting to stay drunk. They talk about judgment against race mixers for violating the Scripture, and they talk about judgment against the world system for rejecting the Lord’s Christ (Ps. 2:2): they are “judgmental.” \cite{Ruckman1992Psalms}}

[32] \footnote{The Old Testament saint has the law in his heart (vs. 31) as well as his head. The righteous man in the Tribulation is not “left” in the hand of the Antichrist, and God will not condemn the man if he is judged by the world court and executed (Rev. 6:9–12). Still, some of the “righteous” are SLAIN (vs. 32), which is apparent from Daniel 11:33. Their case is like the case of the real “righteous” man—the Lord Jesus Christ. They are killed but resurrected (Rev. 6:9–12) for to reign (Rev. 20:1–4). Spiritually speaking, we can use verse 33 with Zechariah 3:1–5 as a picture of what happens when the prosecuting attorney (Rev. 12:9–10) tangles with our Advocate (“daysman,” Job 9:33) in Heaven. \cite{Ruckman1992Psalms}}

[35] \footnote{Consider the prophetic ``tree'' in the parable in Mattehw 13:31--32.}\footnote{Verse 35 is the Antichrist himself, presenting himself in full view and displaying himself, yet he goes completely undetected by Kroll, Jamieson, Fausset, Brown, Yates, Briggs, Duhm, Hitzig, Davidson, Hupfeld, Baethgen, and Delitzsch. They blink at him like Eve blinked at the serpent and pretend that nothing is going on. he A.D. Septuagint—written one hundred years after the New Testament was completed—says the “green bay tree” is a CEDAR. The NKJV, drawn off with the corrupt RV, ASV, NASV, NRSV, and similar publications, has a “native” tree instead of a “bay tree,” and the ASV has just a “green tree.” Since “the Hebrew” (אזרח) refers to a tree that grows on its own soil, the word “bay” as a species has been removed. Here is one of those nezem (“earring” in Gen. 24:22, 30) and “corn” (Mark 2:23) cases where we are bound to find something in the English that has been obliterated by the Hebrew scholars. Now, if the Authorized text says “green bay tree” then a green bay tree it is. There are many trees that are “indigenous” and grow in their “native soil” that would not match the requirements for the “wicked man” at all, but a green bay tree has something to it. Consider \cite{Ruckman1992Psalms}:
\begin{compactenum}
\item The plant is numbered among the Portugal laurels and is related to a sub-family of the rose tribe, but this particular tree is of the laurel variety.
\item The laurel wreath used in all pagan ceremonies (and for the Roman “conquerors”) came from an oak tree.
\item The bay tree, as the oak, is a spreading tree that goes up to about thirty feet, but spreads out much further than that.
\item It is slow in growing and is firmly rooted (as the oak), but it bears no fruit.
\item Like the “live oak” in the South, it is “evergreen” in the winter.
\end{compactenum} These truths match the requirement of the text. The Son of Perdition is slow in coming up, he is unfruitful, but he appears to prosper (see Ps. 73 and Job 21:7-–15) while others suffer, and he spreads abroad until he controls the whole earth (Rev. 13:7). The similitude in the Authorized text revealed something concealed to a Hebrew scholar if he went by “the original Hebrew.”  A typical advanced revelation found in the AV text—which, according to all of the Hebrew scholars, is missing from ALL of the Hebrew texts—is that the wicked one (2 Thess. 2:8) who is a perfect imitation of all that is good and helpful, while at the same time he is connected with Apollyon (Apollos), the god of the underworld, and Baal, the sun god. This information is found only in the AV text, which Bob Jones University calls “King James Onlyism.” That is what was hidden in the AV text so that all the Hebrew scholars missed it. \cite{Ruckman1992Psalms}}
[36] \textcolor[rgb]{0.00,0.00,1.00}{Yet he passed away, and, lo, he \emph{was} not: yea, I sought him, but he could not be found.}
[36] \footnote{Look at the wording in Isaiah 14:6–7 and marvel again at the monstrous stupidity of the Scholars’ Union, including every saved fundamental, premillennial Conservative who swore by “plenary, verbal inspiration.” \cite{Ruckman1992Psalms} Consider some things about this tree:\begin{compactenum}
\item \emph{Laurus nobilis}: the tree is pest and disease resistant (good).
\item The leaves and berries can be used for medicine (good).
\item The old herbalists saw the bay as a virtuous tree (see 2 Cor. 11:15).
\item The bark could be used as a diuretic and for liver ailments (good).
\item “It is a tree of the SUN...under the celestial sign Leo” (the LION).
\item It signified protection and was present at all weddings and funerals.
\item The tree was dedicated to Apollo (moon rocket in Rev. 9:11).
\item The Delphic priests had bay leaves in their mouths when they prophesied.
\item It could have drugged them while they gave out the “oracles.”
\item “BACCALAUREATE” is from the word (baccalaureus) because the academic scholar (Westcott and Hort, Robertson, Thayer, Vincent, Milligan, Trench, et al.) was CROWNED with a laurel wreath.
\end{compactenum} And there it is. Satan, pictured as well as you ever saw him in your life: using healing as “miracles” (Rev. 16:14; 2 Thess. 2:9–10), being connected with the sun god (Baal) and Apollos, having a reputation for being helpful (Gen. 3:4) and “sharing and caring,” posing as a blessing to mankind (2 Pet. 2:19; 2 Cor. 11:13–16), and drugging sinners into trusting a TREE instead of the one who hung on it.}

[40] \footnote{“The wicked” in verses 38 and 40 is a reference to the Antichrist and his followers. The “perfect” and “upright” man who is “righteous” (vss. 37, 39) in “the time of trouble” is the saint of Daniel 11:35 who resists the Antichrist and “endureth to the end” (Matt. 10:22). Of course we can spiritualize since every verse has three applications. The “wicked” can be like Cain, who watched Abel to slay him (vs. 32). \cite{Ruckman1992Psalms}}

